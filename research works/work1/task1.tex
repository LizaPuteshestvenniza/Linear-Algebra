\section{Задание 1}
\plan
\begin{frame}{Условия задачи}
 \[ A = \left(\begin{array}{lll}
    0 & 1 & 0   \\
    -4 & 4 & 0  \\ 
    -2 & 1 & 2 
    \end{array}\right),\quad
    AX = \left(\begin{array}{l}
    1 \\
    -1 \\
    1   
    \end{array}\right).\]
    \begin{block}{План}
        \begin{itemize}
            \item Применить линейное преобразование матрицы A=3x3 к единичному кубику 
            \item Нарисовать и найти собственные вектора вектора А и их span. Сделать контрольный вектор, чтобы можно его было перемещать, согласно выражению Av=w
            \item Найти методом Крамера вектор Х
        \end{itemize}
    \end{block}
\end{frame}

%%%%%%%%%%%%%%%%

\begin{frame}{Начало рассуждений}
Имеем А - линейное преобразование в в $R^3$\\
В базисе $\vec{e_{i}}$ (i = 1,2,3), А имеет матричное представление:
\[
A=\left(\begin{array}{lll}
    0 & 1 & 0 \\
    -4 & 4 & 0 \\
    -2 & 1 & 2
\end{array}\right)
\]
А действует на вектор из $R^3$: если подействовать А на вектор $\vec{x}$, то получим вектор $A\vec{x}$ (назовем его $\vec{y}$, т.е. $\vec{y} = A\vec{x}$) \\
Вектор $\vec{x}$ можно разложить по базисным векторам: $\vec{x} = x_{1}\vec{e_{1}} + x_{2}\vec{e_{2}} +x_{3}\vec{e_{3}}$, где $x_{1}, x_{2}, x_{3}$ - координаты вектора $\vec{x}$ в базисе $\vec{e_{i}}$ (i = 1,2,3)
\end{frame}

%%%%%%%%%%%%%%%%

\begin{frame}{разложение по базису}
Представляем $\vec{e_{1}}, \vec{e_{2}}, \vec{e_{3}}$ в виде:
\[
\left(\begin{array}{l}
    1 \\
    0 \\
    0
\end{array}\right)
\left(\begin{array}{l}
    0 \\
    1 \\
    0
\end{array}\right)
\left(\begin{array}{l}
    0 \\
    0 \\
    1
\end{array}\right)
\]
Путем несложных вычислений получаем, $\vec{x}$ можно представить в виде:
\[
\left(\begin{array}{l}
    x_{1} \\
    x_{2} \\
    x_{3}
\end{array}\right)
\]
Аналогично, разложив $\vec{y}$ по базису, получаем:
\[
\left(\begin{array}{l}
    y_{1} \\
    y_{2} \\
    y_{3}
\end{array}\right)
\]
\end{frame}

%%%%%%%%%%%%%%%%%%%%

\begin{frame}{}
Мы имеем $\vec{y} = A\vec{x}$, т.е.
\[
\left(\begin{array}{lll}
    0 & 1 & 0 \\
    -4 & 4 & 0 \\
    -2 & 1 & 2
\end{array}\right)
\left(\begin{array}{l}
    x_{1} \\
    x_{2} \\
    x_{3}
\end{array}\right) =
\left(\begin{array}{l}
    y_{1} \\
    y_{2} \\
    y_{3}
\end{array}\right) \Rightarrow
    \left\{\begin{array}{l}
    y_{1} = x_{1}\\
    y_{2} = -4x_{1} + 4x_{2}\\
    y_{3} = -2x_{1} +x_{2} +2x_{3}
    \end{array}\right.
\]
Получает, что любой вектор $\vec{x}$ из $R^3$ с координатами $x_{1}, x_{2}, x_{3}$ в базисе $\vec{e_{i}}$ (i = 1,2,3) А превратит в вектор $\vec{y}$ из $R^3$ с координатами $y_{1}, y_{2}, y_{3}$ в базисе $\vec{e_{i}}$ (i = 1,2,3), где $y_{1} = x_{1}, y_{2} = -4x_{1} + 4= x_{2}, y_{3} = -2= x_{1} += x_{2} +2= x_{3}$ \\
Обычно, смотрят как А действует на базис $\vec{e_{i}}$ (i = 1,2,3), чтоб легко понять, во что А превратит $\vec{x}$
\end{frame}

%%%%%%%%%%%%%%%%%%%

\begin{frame}{Действие $A$ на $\vec{e_{1}}$}
\begin{columns}
\begin{column}{0.5\paperwidth}
$A$ действует на $\vec{e_{1}}$ подобным образом: $A\vec{e_{1}}$ \\
А именно:
\[
\left(\begin{array}{lll}
    0 & 1 & 0  \\
    -4 & 4 & 0  \\ 
    -2 & 1 & 2 
\end{array}\right)
\left(\begin{array}{l}
    1 \\
    0 \\ 
    0 
\end{array}\right) =
\left(\begin{array}{l}
    0 \\
    -4 \\
    -2
\end{array}\right)
\]
То есть $A$ берет и преобразует
\[
\left(\begin{array}{l}
    1 \\
    0 \\ 
    0 
\end{array}\right) \Rightarrow
\left(\begin{array}{l}
    0 \\
    -4 \\ 
    -2 
\end{array}\right)
\]
Получается, что $A$ повернуло и растянуло по длине $\vec{e_{1}}$
\end{column}
\begin{column}{0.3\paperwidth}
\includegraphics[width=1.3\columnwidth]{fig/task-1/first_transform.png}
\end{column}
\end{columns}
\end{frame}

%%%%%%%%%%%%%%%

\begin{frame}{Действие $A$ на $\vec{e_{2}}$}
\begin{columns}
\begin{column}{0.5\paperwidth}
$A$ действует на $\vec{e_{2}}$ подобным образом: $A\vec{e_{2}}$ \\
А именно:
\[
\left(\begin{array}{lll}
    0 & 1 & 0  \\
    -4 & 4 & 0  \\ 
    -2 & 1 & 2 
\end{array}\right)
\left(\begin{array}{l}
    0 \\
    1 \\ 
    0 
\end{array}\right) =
\left(\begin{array}{l}
    1 \\
    4 \\
    1
\end{array}\right)
\]
То есть $A$ берет и преобразует
\[
\left(\begin{array}{l}
    0 \\
    1 \\ 
    0 
\end{array}\right) \Rightarrow
\left(\begin{array}{l}
    1 \\
    4 \\
    1
\end{array}\right)
\]
Получается, что $A$ повернуло и растянуло по длине $\vec{e_{1}}$
\end{column}
\begin{column}{0.2\paperwidth}
\includegraphics[width=1.3\columnwidth]{fig/task-1/second_transform.png}
\end{column}
\end{columns}
\end{frame}

%%%%%%%%%%%%%%%%%

\begin{frame}{Действие $A$ на $\vec{e_{3}}$}
\begin{columns}
\begin{column}{0.5\paperwidth}
$A$ действует на $\vec{e_{3}}$ подобным образом: $A\vec{e_{3}}$ \\
А именно:
\[
\left(\begin{array}{lll}
    0 & 1 & 0  \\
    -4 & 4 & 0  \\ 
    -2 & 1 & 2 
\end{array}\right)
\left(\begin{array}{l}
    0 \\
    0 \\ 
    1
\end{array}\right) =
\left(\begin{array}{l}
    0 \\
    0 \\
    2
\end{array}\right)
\]
То есть $A$ берет и преобразует
\[
\left(\begin{array}{l}
    0 \\
    0 \\ 
    1 
\end{array}\right) \Rightarrow
\left(\begin{array}{l}
    0 \\
    0 \\
    2
\end{array}\right)
\]
Получается, что $A$ растянуло $\vec{e_{3}}$ в 2 раза. 
\end{column}
\begin{column}{0.2\paperwidth}
\includegraphics[width=1.3\columnwidth]{fig/task-1/third_transform.png}
\end{column}
\end{columns}
\end{frame}

%%%%%%%%%%%%%%%%

\begin{frame}{}
Когда мы растягиваем вектор $\vec{v}$ по базису, мы говорим, что в векторе $\vec{v}$ мы продвинулись вдоль $\vec{v_{1}}, \vec{v_{2}},\vec{v_{3}}$ на:
\[
\left(\begin{array}{l}
    1 \\
    0 \\
    0
\end{array}\right),
\left(\begin{array}{l}
    0 \\
    1 \\
    0
\end{array}\right),
\left(\begin{array}{l}
    0 \\
    0 \\
    1
\end{array}\right)
\]
и получили вектор $\vec{v}$ \\
Посчитав $A\vec{v}$, получим: $A\vec{v} = A(v_{1}\vec{e_{1}} + v_{2}\vec{e_{2}} +v_{3}\vec{e_{3}}) = v_{1}A\vec{e_{1}} + v_{2}A\vec{e_{2}} +v_{3}A\vec{e_{3}}$ \\
Назовем вектор $A\vec{v}$ буквой $\vec{u}$ \\
Тогда если мы продвинимся на $v_{1}$ вдоль $A\vec{e_{1}}$, на $v_{2}$ вдоль $A\vec{e_{2}}$, на $v_{3}$ вдоль $A\vec{e_{3}}$, получаем $\vec{u}$\\
Получается, что один раз вычислив $A\vec{e_{1}}, A\vec{e_{2}}, A\vec{e_{3}}$, то всегдо сможем быстро найти $A\vec{v}$
\end{frame}

%%%%%%%%%%%%%%%%

\begin{frame}{Единичный кубик}
\begin{columns}
\begin{column}{0.5\paperwidth}
Рассмотрим все векторы, которые лежат в единичном кубике с вершинами: 
\begin{itemize}
    \item (0,0,0);
    \item (0,0,1);
    \item (0,1,0);
    \item (0,1,1);
    \item (1,0,0);
    \item (1,0,1);
    \item (1,1,0);
    \item (1,1,1)
\end{itemize}
Вектор $\vec{v}$ в этом кубике записывается так: $\vec{v} = a\vec{e_{1}}+b\vec{e_{2}}+c\vec{e_{3}}$, где $a,b,c \in [0, 1]$
\end{column}
\begin{column}{0.4\paperwidth}
\includegraphics[width=1\columnwidth]{fig/task-1/Единичный_кубик.png}
\end{column}
\end{columns}   
\end{frame}

%%%%%%%%%%%%%%%%

\begin{frame}{Пример}
\begin{columns}
\begin{column}{0.5\paperwidth}
\begin{itemize}
    \item $\vec{v} = {1,0,0}$, $a=1,b=0,c=0$
    \item $\vec{v} = {\frac{1}{2},\frac{1}{2},\frac{1}{2}}$, $a=\frac{1}{2},b=\frac{1}{2},c=\frac{1}{2}$
    \item $\vec{v} = {1,0,1}$, $a=1,b=0,c=1$
\end{itemize}
\begin{center}
\includegraphics[width=0.7\columnwidth]{fig/task-1/Пример3.png}  
\small Номер 3
\end{center}
\end{column}
\begin{column}{0.3\paperwidth}
\includegraphics[width=1\columnwidth]{fig/task-1/Пример1.png}
\small Номер 1
\includegraphics[width=1\columnwidth]{fig/task-1/Пример2.png}
\small Номер 2
\end{column}
\end{columns}   
\end{frame}

%%%%%%%%%%%%%%%%%

\begin{frame}{Получение параллелограмма}
Меняя независимо a, b, c в пределах от 0 до 1 будем получать вектор с концом внутри кубика (или на границе кубика) \\
А изменит каждый из векторов внутри и на кубике таким образом: $A\vec{v}$ \\
Тогда куб превратиться в: $A\vec{v} = A(a\vec{e_{1}} + b\vec{e_{2}} +c\vec{e_{3}}) = aA\vec{e_{1}} + bA\vec{e_{2}} +cA\vec{e_{3}}$ \\
Получается, любой вектор $\vec{v}$ из кубика, где мы продвинулись на a, b , c вдоль:
\[
\left(\begin{array}{l}
    1 \\
    0 \\
    0
\end{array}\right)
\left(\begin{array}{l}
    0 \\
    1 \\
    0
\end{array}\right)
\left(\begin{array}{l}
    0 \\
    0 \\
    1
\end{array}\right)
\]
Станет вектором из новой области, где мы продвинулись на a,b,c на $A\vec{e_{1}}, A\vec{e_{2}},A\vec{e_{3}},$
\begin{center}
\includegraphics[width=0.3\columnwidth]{fig/task-1/ПервыйПараллелограмм.png} 
\end{center}
\end{frame}

%%%%%%%%%%%%%%

\begin{frame}
Получаем параллелограмм с вершинами: (0,0,2), (1,4,3), (1,4,1), (1,0,-1), (1,0,1), (0,-4,0), (0,-4,-2), (0,0,0)\\
Каждый вектор из исходного кубика после применения А остается внутри этого параллелограмма (каждый вектор как-то повернется и растянется)\\
Можно сказать, что наш кубик повернется и растянется. \\
С помощью формулы:
\[
A \cdot \vec{v}:
\vec{v_{1}}\left(\begin{array}{l}
    0 \\
    -4 \\
    -2
\end{array}\right) +
\vec{v_{2}}\left(\begin{array}{l}
    1 \\
    4 \\
    1
\end{array}\right) +
\vec{v_{3}}\left(\begin{array}{l}
    0 \\
    0 \\
    2
\end{array}\right) +
\]
можно сказать какие вершины куда переместились. 
\end{frame}

%%%%%%%%%%%%%%%%

\begin{frame}{Вершины параллелограмма}
\begin{columns}
\begin{column}{0.4\paperwidth}
Перенос вершин:
\begin{itemize}
    \item (0,0,0) $\Rightarrow$ (0,0,0);
    \item (0,0,1) $\Rightarrow$ (0,0,2)
    \item (0,1,0) $\Rightarrow$ (1,4,1);
    \item (0,1,1) $\Rightarrow$ (1,4,3);
    \item (1,0,0) $\Rightarrow$ (0,-4,-2);
    \item (1,0,1) $\Rightarrow$ (0,-4,0);
    \item (1,1,0) $\Rightarrow$ (1,0,-1);
    \item (1,1,1) $\Rightarrow$ (1,0,1);
  \end{itemize}
\end{column}
\begin{column}{0.5\paperwidth}
\includegraphics[width=1\columnwidth]{fig/task-1/Параллелограмм.png}
\end{column}
\end{columns}
\end{frame}

%%%%%%%%%%%%%%%

\begin{frame}{Понятие собственного вектора и числа}
Собственный вектор матрицы $A$ - это вектор, который после применения этой матрицы не поворачивается, а только растягивается (или сжимается) на число $\lambda$ (собственное число) \\

Берем вектор $\vec{v}$ и применяем к нему $A$: $A\vec{v}$, в результате получааем растянутый (или сжатый) на $\lambda$ исходный вектор (То есть $\lambda\vec{v}$). Таким образом:  $A\vec{v} = \lambda\vec{v}$
\end{frame}

%%%%%%%%%%%%%%%%%%%%

\begin{frame}{Единичная матрица}
Единичная матрица не меняет вектор: $E\vec{v} = \vec{v}$ \\
$$ A\vec{v} = \lambda\vec{v} \Rightarrow  A\vec{v} = \lambda E\vec{v} \Rightarrow A\vec{v} - \lambda E\vec{v} = \vec{0} \Rightarrow (A - \lambda E)\vec{v} = \vec{0} $$ \\
\[
A=\left(\begin{array}{lll}
    0 & 1 & 0 \\
    -4 & 4 & 0 \\
    -2 & 1 & 2
    \end{array}\right),
E=\left(\begin{array}{lll}
    1 & 0 & 0 \\
    0 & 1 & 0 \\
    0 & 0 & 1
    \end{array}\right) 
\]
\[
A - \lambda E =\left(\begin{array}{lll}
    0 & 1 & 0 \\
    -4 & 4 & 0 \\
    -2 & 1 & 2
    \end{array}\right) -
\lambda \left(\begin{array}{lll}
    1 & 0 & 0 \\
    0 & 1 & 0 \\
    0 & 0 & 1
    \end{array}\right) =
\left(\begin{array}{lll}
    0 & 1 & 0 \\
    -4 & 4 & 0 \\
    -2 & 1 & 2
    \end{array}\right) -
\left(\begin{array}{lll}
    \lambda & 0 & 0 \\
    0 & \lambda & 0 \\
    0 & 0 & \lambda
    \end{array}\right) =
\]
\[
\left(\begin{array}{lll}
    -\lambda & 1 & 0 \\
    -4 & 4 - \lambda & 0 \\
    -2 & 1 & 2 - \lambda
    \end{array}\right)   
\]
\end{frame}

%%%%%%%%%%%%%%%

\begin{frame}{Нахождение собственного числа}
Посчитаем определитеь для $A-\lambda E$:
\[
\det(A-\lambda E) = \left|\begin{array}{ccc}
        -\lambda & 1 & 0 \\
        -4 & 4-\lambda & 0 \\
        -2 & 1 & 2-\lambda
        \end{array}\right|=(-\lambda(\lambda - 4)-4)(\lambda -2)
\]
Решим уравнение: $\det(A-\lambda E)=0$ \\
$(-\lambda(\lambda - 4)-4)(\lambda -2) =0 $\\
$(\lambda-2)^{3}=0$ \\
Получаем: $\lambda=2$ - собственное число (вырождено, трекратно)
\end{frame}

\begin{frame}{Получение $\vec{v_{1}}, \vec{v_{2}}, \vec{v_{3}}$}
Собственному числу $\lambda$ соответствует собственный вектор $\vec{v}: A\vec{v} = 2\vec{v}$  \\
Найдем $\vec{v}:$
\[
\vec{v}=\left(\begin{array}{lll}
    0 & 1 & 0 \\
    -4 & 4 & 0 \\
    -2 & 1 & 2
\end{array}\right)
\left(\begin{array}{l}
    \vec{v_{1}} \\
    \vec{v_{2}} \\
    \vec{v_{3}} 
\end{array}\right) =
2 \cdot \left(\begin{array}{l}
    \vec{v_{1}} \\
    \vec{v_{2}} \\
    \vec{v_{3}} 
\end{array}\right) 
\]
Получаем:
$$
    \left\{\begin{array}{l}
    \vec{v_{2}} =  \vec{v_{1}} \\
    -2\vec{v_{1}} +  \vec{v_{2}} = 0 \\
    -2\vec{v_{1}} +  \vec{v_{2}} = 0
    \end{array}\right.
$$
Отсюда видно, что из вектора $\vec{v}$ может быть любое значение $\vec{v_{3}}$, а $\vec{v_{1}}$ и $\vec{v_{2}}$ связаны $2\vec{v_{1}} = \vec{v_{2}}$
\end{frame}

%%%%%%%%%%%%%%%

\begin{frame}{Вычисление первого собственного вектора $\vec{v}$}
Предположим, что $\vec{v_{3}} = 1$, а $\vec{v_{1}} = 0$, тогда $\vec{v_{2}} = 0$, и получаем:
\[
\vec{v}=\left(\begin{array}{l}
    0 \\
    0 \\
    1
\end{array}\right)
\]
Так как уже наблюдалось, что:
\[
A \cdot \left(\begin{array}{l}
    0 \\
    0 \\
    1
\end{array}\right) = 
\left(\begin{array}{l}
    0 \\
    0 \\
    2
\end{array}\right) = 
2 \cdot \left(\begin{array}{l}
    0 \\
    0 \\
    1
\end{array}\right)
\]
Значит $\vec{v}$ - собственный вектор $A$ с собственным числом 2. 
\end{frame}

%%%%%%%%%%%%%%%%%%%%

\begin{frame}{Вычисление второго собственного вектора $\vec{v}$}
\begin{columns}
\begin{column}{0.5\paperwidth}
Путем несложных вычислений, получаем второй собственный вектор $A$ с собственным числом 2
\[
\vec{v}=\left(\begin{array}{l}
    1 \\
    2 \\
    0
\end{array}\right)
\]
Третий собственный вектор линейно независимый от этих двух, найти его не удастся. 

Span этих двух векторов - плоскость, образованная этими двумя векторами, т.е. все векторы, которые описываются $\vec{v} = a \cdot \vec{v_{1}} + b \cdot \vec{v_{2}}  a,b \in R$
\end{column}
\begin{column}{0.4\paperwidth}
\includegraphics[width=1\columnwidth]{fig/task-1/Полученная_Плоскость.png}
\end{column}
\end{columns}
\end{frame}

%%%%%%%%%%%%%%%%%%%%

\begin{frame}
\begin{columns}
\begin{column}{0.5\paperwidth}
\begin{center}
\href{https://www.geogebra.org/calculator/hwswsnrm}{Переход вектора после преобразования матрицы}
\end{center}
\end{column}
\begin{column}{0.5\paperwidth}
\includegraphics[width=0.9\columnwidth]{fig/task-1/Вектора.png} 
\end{column}
\end{columns}
\end{frame}

%%%%%%%%%%%%%%%%%%%%%

\begin{frame} {Метод Крамера}
\begin{multline*}
x=\frac{\det\left(\begin{array}{lll} 
1 & 1 & 0 \\
-1 & 4 & 0\\
1 & 1 & 2
\end{array}\right)}{\det\left(\begin{array}{lll} 
0 & 1 & 0 \\
-4 & 4 & 0\\
-2 & 1 & 2
\end{array}\right)}
=\frac{(-1)^{1+1}\cdot(4\cdot2 -1\cdot 0)-(1\cdot 2 - 1\cdot 0)+(1\cdot 0-4\cdot 0)}{0\cdot(4\cdot 2-1\cdot 0)-(-4)\cdot(1\cdot2-1\cdot0) -2\cdot(1\cdot 0 -4 \cdot 0)}=\\=\dfrac{10}{8}=1.25
\end{multline*}
\end{frame}
\begin{frame} {Метод Крамера}

\begin{multline*}
y=\frac{\det\left(\begin{array}{lll} 
0 & 1 & 0 \\
4 & -1 & 0\\
2 & 1 & 2
\end{array}\right)}{\det\left(\begin{array}{lll} 
0 & 1 & 0 \\
-4 & 4 & 0\\
-2 & 1 & 2
\end{array}\right)}
=\frac{0\cdot(-1\cdot 2 - 1 \cdot 0)-4\cdot(1\cdot2 -1\cdot0)-2\cdot(1\cdot 0+0)}{0\cdot(4\cdot 2-1\cdot 0)-(-4)\cdot(1\cdot2-1\cdot0) -2\cdot(1\cdot 0 -4 \cdot 0)}=\\=\dfrac{8}{8}=1\\
\end{multline*}
\end{frame}

\begin{frame} {Метод Крамера}
\begin{multline*}
z=\frac{\det\left(\begin{array}{lll} 
0 & 1 & 1 \\
-4 & 4 & -1\\
-2 & 1 & 1
\end{array}\right)}{\det\left(\begin{array}{lll} 
0 & 1 & 0 \\
-4 & 4 & 0\\
-2 & 1 & 2
\end{array}\right)}
=\frac{0\cdot(4\cdot 1 + 1 \cdot 1)-4\cdot(1\cdot1 -1\cdot1)-2\cdot(1\cdot(-1)-4 \cdot 1)}{0\cdot(4\cdot 2-1\cdot 0)-(-4)\cdot(1\cdot2-1\cdot0) -2\cdot(1\cdot 0 -4 \cdot 0)}=\\=\dfrac{10}{8}=1.25\\
\end{multline*}
\end{frame}


