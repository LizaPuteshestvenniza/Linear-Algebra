\documentclass[aspectratio=169]{beamer}

\usepackage[utf8]{inputenc}
\usepackage[T1,T2A]{fontenc}
\usepackage[english,russian]{babel}
\usepackage{graphicx}
\usepackage{
amsmath, % базовый пакет для отображения математический формул
amssymb, % дополнительные математический символы
polynom % позволяет удобно работать с многочленами
}
\usepackage{cases}

% Без этой команды он иногда ругается.
\hypersetup{unicode=true}

% Чтобы адекватно работало копирование текста из полученной .pdf-ки.
% \usepackage{cmap}

% Это нужно, чтобы он называл рисунки без сокращения "рис.". Таблицы он называет без сокращения по умолчанию.
\addto\captionsrussian{\renewcommand{\figurename}{Рисунок}}

% Пакет для использования запятой в качестве десятичного разделителя. Следите, чтобы в формулах запятые стояли с пробелами, там где они запятые. Например $v = (x, y, z)$
% \usepackage{icomma}

% Это единственный пакет для библиографии, который у меня заработал с \footcite шаблоном.
% В презентациях лучше делать её руками через \footnote!
% \usepackage[style=mla]{biblatex}
% \addbibresource{references.bib}

\usepackage{ITMOtheme}

% Рисунок для титульного слайда.
\titlegraphic{\includegraphics[scale=.8]{itmo/logo_rus_vert_blue.pdf}}



% Поля title, author, subject, keywords используются при формировании pdf документа. Поэтому их нужно заполнять, даже если вы формируете титульный слайд руками.
% Формат: \title[Короткое название]{Полное название}
\title[Типовик №1]{Типовой расчет №1}
\subtitle[Линейная алгебра]{Линейная алгебра}

% \institute[ИТМО]{Национальный исследовательский университет информационных технологий, механики и оптики}

\where{Санкт-Петербург}

% Тематика и ключевые слова.
% \subject{Beamer template}
% \keywords{ITMO University, LaTex teamplate, beamer}

% По умолчанию внизу каждого слайда пишется название презентации (\inserttitle). Этот текст можно заменить на другой, например:
\setfootlinetext{\insertsection}

% Кастомная команда, выводящая позицию в презентации
\newcommand{\plan}
{
    \begin{frame}[plain]{Этапы работы}
        % \small
        \tableofcontents[currentsection]
    \end{frame}
}


\begin{document}

% [plain] - модификатор для создания пустого слайда (без нижней полосы). Идеально подходит для создания первого (титульного) и последнего слайда с полигональным фоном, либо для переходных слайдов между главами или слайдов с оглавлением.
% \itmopolygons - команда для создания полигонального фона. Используется для создания фона на первом (титульном) и последнем слайде.

\begin{frame}[plain]
    \titlepage
\end{frame}

\begin{frame}[plain]{Содержание}
    % \small
    \tableofcontents
\end{frame}

\section{Задние 1}
\subsection{Формулировка задания}
\begin{frame}{Наибольшее и наименьшее значение функции\\
нескольких переменных в области} %это заголовок слайда

\begin{block}{Формулировка задания}
Через точку $D$ $(a,b,c)$ проведите плоскость так, чтобы объём тетраэдра, отсекаемого ею от
координатного трёхгранника, был бы наименьшим. Изобразите на графике для конкретной 
точки $А$.
\end{block}

\begin{block}{План:}
\begin{itemize}
    \item Изобразим на рисунке условие задачи
    \item Решим задачу аналитически
    \item Проиллюстрируем ответ
\end{itemize}
\end{block}
\end{frame}

\subsection{Выполнение задания}
\begin{frame}{Выведение формулы объема}
\begin{columns}[T]
    \begin{column}{0.48\textwidth}
    \begin{figure}
        \centering
        \includegraphics[scale=0.35]{images/tetrader.png}
        \caption{Отсеченная пирамида}
        \label{fig:pyramid}
    \end{figure}
    \end{column}
    \begin{column}{0.48\textwidth}
        Уравнение искомой плоскости: $\frac{x}{A}+\frac{y}{B}+\frac{z}{C}=1$, где $A, B$ и $C$ являются отрезками, отсекаемыми плоскостью на координатных осях. Поскольку точка $D$ $(a,b,c)$ принадлежит плоскости, то тогда: $$\frac{a}{A}+\frac{b}{B}+\frac{c}{C}=1 \Rightarrow C=\frac{c}{1-\frac{a}{A}-\frac{b}{B}}$$
        Объем тетраэдра: $$V=\frac{1}{6} A B C=\frac{1}{6} A B \frac{c}{1-\frac{a}{A}-\frac{b}{B}}$$
    \end{column}
\end{columns}
\end{frame}

\begin{frame}{Выведени уравнения плоскости}
$$
\frac{\partial V}{\partial A}=\frac{B c}{6} \frac{1-\frac{2 a}{A}-\frac{b}{B}}{\left(1-\frac{a}{A}-\frac{b}{B}\right)^{2}}; \quad\frac{\partial V}{\partial B}=\frac{A c}{6} \frac{1-\frac{a}{A}-\frac{2 b}{B}}{\left(1-\frac{a}{A}-\frac{b}{B}\right)^{2}}
$$
\Large
$$
\left\{\begin{array} { l } 
{ \frac { \partial V } { \partial A } = 0 , } \\
{ \frac { \partial V } { \partial B } = 0 }
\end{array} \Rightarrow \left\{\begin{array} { c } 
{ 1 - \frac { 2 a } { A } - \frac { b } { B } = 0 , } \\
{ 1 - \frac { a } { A } - \frac { 2 b } { B } = 0 }
\end{array} \Rightarrow \left\{\begin{array}{c}
A=3 a \\
B=3 b \\
C=3 c
\end{array}\right.\right.\right.
$$
Таким образом, получаем уравнение плоскости:
$$\frac{x}{a}+\frac{y}{b}+\frac{z}{c}=3 .$$
\end{frame}

\begin{frame}{Пример для точки E(3,2,1)}
\begin{columns}[T]
    \begin{column}{0.48\textwidth}
    \begin{figure}
        \centering
        \includegraphics[scale=0.4]{images/piramidD.png}
        \caption{Отсеченная пирамида}
        \label{fig:pyramid_E_point}
    \end{figure}
    \end{column}
    \begin{column}{0.48\textwidth}
       Получаем уравнение плоскости:
       $$\frac{x}{3}+\frac{y}{2}+\frac{z}{1}=3$$
    \end{column}
\end{columns}
\end{frame}


\section{Задние 2}
\subsection{Формулировка задания}
\begin{frame}{Интегралы Пуассона и Френеля}  %это заголовок слайда
\begin{block}{Формулировка задания}
Вычислите интеграл:
$$
\int_{0}^{\infty} \frac{\cos (4 t-\pi / 2)}{\sqrt{t}} d t
$$
\end{block}
\begin{block}{План:}
\begin{itemize}
    \item Вычислим Гауссов интеграл и его квадрат
    \item Вычислим интеграл $\int_{0}^{\infty} \frac{\sin t}{\sqrt{t}} d t=J$
    \item С помощью предыдущего интеграла вычислим интеграл $K$ и интегралы $\int_{0}^{\infty} \sin x^{2} d x$ и $\int_{0}^{\infty} \sin \left(\pi x^{2} / 2\right) d x$
    \item Проиллюстрируем графики функции ошибок, интегралов Френеля и их подынтегральных функций
\end{itemize}
\end{block}
\end{frame}

\subsection{Выполнение задания}
\begin{frame}{Вычислим Гауссов интеграл}
\begin{align*}
&\text { Заметим, что } I=\int_{0}^{\infty} e^{-x^{2}} d x=\int_{0}^{\infty} e^{-y^{2}} d y .\\
&\text { Тогда } I^{2}=\int_{0}^{\infty} e^{-x^{2}} d x \int_{0}^{\infty} e^{-y^{2}} d y-\text { двукратный интеграл. }\\
&I^{2}=\int_{0}^{\infty} e^{-x^{2}} d x \int_{0}^{\infty} e^{-y^{2}} d y=\int_{0}^{\infty} \int_{0}^{\infty} e^{-\left(x^{2}+y^{2}\right)} d x d y
\end{align*}
\end{frame}

\begin{frame}{Вычислим Гауссов интеграл}
Сделаем переход в полярную систему координат:
$$
\begin{array}{l}
d S=d x d y=r d \varphi \cdot d r \\
x=r \cos \varphi \\
y=r \sin \varphi
\end{array} \mid \rightarrow x^{2} \cos ^{2} \varphi+y^{2} \sin ^{2} \varphi=r^{2} \rightarrow x^{2}+y^{2}=r^{2}
$$
\begin{align*}
I^{2}&=\int_{0}^{\infty} \int_{0}^{\infty} e^{-\left(x^{2}+y^{2}\right)} d x d y=\int_{0}^{\frac{\pi}{2}} \int_{0}^{\infty} e^{-r^{2}} r d \varphi d r=\int_{0}^{\frac{\pi}{2}} d \varphi \int_{0}^{\infty} e^{-r^{2}} r d r=\\
&=\int_{0}^{\frac{\pi}{2}} d \varphi \int_{0}^{\infty} e^{-r^{2}} \frac{1}{2} d\left(r^{2}\right)=\frac{1}{2} \int_{0}^{\frac{\pi}{2}} d \varphi\left(-\left.e^{-r^{2}}\right|_{0} ^{\infty}\right)=\\
&=\frac{1}{2} \int_{0}^{\frac{\pi}{2}} d \varphi\left(-e^{-\infty}-\left(-e^{0}\right)\right)=\frac{1}{2} \int_{0}^{\frac{\pi}{2}} d \varphi\left(-\left.e^{-r^{2}}\right|_{0} ^{\infty}\right)=\\
&=\frac{1}{2} \int_{0}^{\frac{\pi}{2}} d \varphi\left(-e^{-\infty}-\left(-e^{0}\right)\right)=\frac{1}{2} \int_{0}^{\frac{\pi}{2}} d \varphi=\frac{1}{2}\left(\left.\varphi\right|_{0} ^{\frac{\pi}{2}}\right)=\frac{\pi}{4}
\end{align*}
\end{frame}

\begin{frame}{Доказательство полезного тождества}
С учетом небольшой замены, легко увидеть, что:
$$
\int_{0}^{\infty} e^{-n x^{2}} d x=\left[\begin{array}{l}
p=\sqrt{n} x \\
p^{2}=n x^{2} \\
\frac{d p}{\sqrt{n}}=d x
\end{array}\right]=\frac{1}{\sqrt{n}} \int_{0}^{\infty} e^{-p^{2}} d p=\frac{1}{\sqrt{n}} I
$$
На предыдущем слайде мы получили, что $I = \frac{\sqrt{\pi}}{2}$ Тогда справедливо и равенство:
$$
\frac{1}{\sqrt{t}}=\frac{2}{\sqrt{\pi}} \int_{0}^{\infty} e^{-u^{2} t} d u
$$
Данное тождество будет полезно нам вдальнейшим.
\end{frame}
\begin{frame}{Вычисление интеграла J}
$$
J=\int_{0}^{\infty} \frac{\sin t}{\sqrt{t}} d t
$$
$$
\frac{1}{\sqrt{t}}=\frac{2}{\sqrt{\pi}} \int_{0}^{\infty} e^{-u^{2} t} d u \Rightarrow J=\frac{2}{\sqrt{\pi}} \int_{0}^{\infty} d t \int_{0}^{\infty} d u e^{-t u^{2}}\sin{t}
$$
Теперь возьмем интеграл по $t$, обозначив подынтегральную функцию как $Q\left(u^{2}\right) .$ Тогда:
$$
Q(a)=\int_{0}^{\infty} e^{-a t} \sin t d t=\frac{1}{a^{2}+1}
$$
Тем самым, получаем следующий интеграл:
$$J=\frac{2}{\sqrt{\pi}} \int_{0}^{\infty} \frac{1}{1+u^{4}} d u$$
Сделаем замену $u=\frac{1}{t}$
\end{frame}
\begin{frame}{Вычисление интеграла J}
$$
J=\frac{2}{\sqrt{\pi}} \int_{0}^{\infty}\left(-\frac{d t}{t^{2}}\right) \frac{1}{1+1 / t^{4}}=\frac{2}{\sqrt{\pi}} \int_{0}^{\infty} \frac{t^2}{1+t^{4}} d t=
$$
$$
J=\frac{2}{\sqrt{\pi}} \int_{0}^{\infty} \frac{t^2}{1+t^{4}} d t
$$
Беря полусумму двух представлений для интеграла I, получим:
$$
J=\frac{1}{\sqrt{\pi}} \int_{0}^{\infty} \frac{u^2+1}{1+u^{4}} d t
$$
Теперь можно перейти к стандарнтой переменной для интегрирования \\
симметрических многочленов $t=x-\frac{1}{x} ;$ при этом $d t=\left(1+\frac{1}{x^{2}}\right) d x$, получим:
$$
I=\frac{1}{\sqrt{\pi}} \int_{-\infty}^{\infty} \frac{d t}{t^{2}+2}=\left.\frac{1}{2 \sqrt{2 \pi}} \arctan \frac{t}{\sqrt{2}}\right|_{-\infty} ^{\infty}=\sqrt{\frac{\pi}{2}}
$$
\end{frame}

\begin{frame}{Вычисление интеграла K}
$$K=\int_{0}^{\infty} \frac{\cos (4 t-\pi / 2)}{\sqrt{t}} d t=\int_{0}^{\infty} \frac{\sin(4t)}{\sqrt{t}}dt$$
Сделаем замену $k=4t$, $dk=4dt$:
$$K=\int_{0}^{\infty} \frac{\sin(k)}{\sqrt{\frac{k}{4}}}\frac{dk}{4}=\int_{0}^{\infty} \frac{2\sin(k)}{\sqrt{k}}\frac{dk}{4}=\int_{0}^{\infty} \frac{\sin(k)}{2\sqrt{k}}dk=\frac{\sqrt{\pi}}{2^{1.5}}$$
\end{frame}

\begin{frame}{Пример вычисления другого интеграла}
$$\int_{0}^{\infty} \sin x^{2} d x$$
Перейдем к переменной $t=x^2$:
$$\int_{0}^{\infty} \frac{\sin t}{2\sqrt{t}}dt$$
$$
\frac{1}{\sqrt{t}}=\frac{2}{\sqrt{\pi}} \int_{0}^{\infty} e^{-x^{2} t} d x \Rightarrow I=\frac{1}{\sqrt{\pi}} \int_{0}^{\infty} d t \int_{0}^{\infty} d x e^{-t x^{2}} \sin t
$$
Теперь возьмем интеграл по $t$, обозначив подынтегральную функцию как $J\left(x^{2}\right)$. Тогда:
$$
J(a)=\int_{0}^{\infty} e^{-\alpha t} \sin t d t=\frac{1}{a^{2}+1}
$$
\end{frame}

\begin{frame}{Пример вычисления другого интеграла}
Тем самым, получаем следующий интеграл:
$$
I=\frac{1}{\sqrt{\pi}} \int_{0}^{\infty} \frac{1}{1+x^{4}} d x
$$
Сделаем замену $x=\frac{1}{t}$
$$
I=\frac{1}{\sqrt{\pi}} \int_{0}^{\infty}\left(-\frac{d t}{t^{2}}\right) \frac{1}{1+1 / t^{4}}=\frac{2}{\sqrt{\pi}} \int_{0}^{\infty} \frac{t^2}{1+t^{4}} d t=
$$
$$
I=\frac{2}{\sqrt{\pi}} \int_{0}^{\infty} \frac{t^2}{1+t^{4}} d t
$$
Беря полусумму двух представлений для интеграла I, получим:
$$
I=\frac{1}{2\sqrt{\pi}} \int_{0}^{\infty} \frac{x^2+1}{1+x^{4}} d t
$$
\end{frame}

\begin{frame}{Пример вычисления другого интеграла}
Теперь можно перейти к стандарнтой переменной для интегрирования \\
симметрических многочленов $t=x-\frac{1}{x} ;$ при этом $d t=\left(1+\frac{1}{x^{2}}\right) d x$, получим:
$$
I=\frac{1}{2\sqrt{\pi}} \int_{-\infty}^{\infty} \frac{d t}{t^{2}+2}=\left.\frac{1}{4 \sqrt{2 \pi}} \arctan \frac{t}{\sqrt{2}}\right|_{-\infty} ^{\infty}=\frac{\sqrt{\pi}}{2^{1.5}}
$$
\end{frame}

\begin{frame}{Пример вычисления другого интеграла}
$$
\int_{0}^{\infty} \sin \left(\pi x^{2} / 2\right) d x
$$
Пусть: $k^2=\pi x^{2} / 2$, тогда $dk=\frac{\sqrt{2}\pi}{2\sqrt{\pi}}dx$, $dx=\frac{2\sqrt{\pi}dk}{\sqrt{2}\pi}$
$$
\int_{0}^{\infty} \sin \left(k^2\right) \frac{2\sqrt{\pi}dk}{\sqrt{2}\pi}=\frac{2\sqrt{\pi}dk}{\sqrt{2}\pi}\int_{0}^{\infty} \sin \left(k^2\right)dk=\frac{2\sqrt{\pi}}{\sqrt{2}\pi}*\frac{\sqrt{\pi}}{2^{1.5}}=\frac{1}{2}
$$
\end{frame}

\begin{frame}{График функции ошибок}
    \begin{figure}
        \centering
        \includegraphics[scale=0.67]{images/Error_Function.svg.png}
    \end{figure}
\end{frame}
\begin{frame}{Графики интегралов Френеля}
    \begin{figure}
        \centering
        \includegraphics[scale=0.22]{images/Fresnel_Integrals.svg.png}
    \end{figure}
\end{frame}
\begin{frame}{Графики подынтыгральных выражений Френеля}
    \begin{columns}[T]
        \begin{column}{0.48\textwidth}
        \begin{figure}
            \centering
            \includegraphics[scale=0.32]{images/sin(x^2).png}
            \caption{$f(x)=sin(x^2)$}
            \label{fig:sin(x^2)}
        \end{figure}
        \end{column}
        \begin{column}{0.48\textwidth}
        \begin{figure}
            \centering
            \includegraphics[scale=0.32]{images/cos(x^2).png}
            \caption{$f(x)=cos(x^2)$}
            \label{fig:cos(x^2)}
        \end{figure}
        \end{column}
    \end{columns}
\end{frame}

% \include{example}

\begin{frame}[plain,noframenumbering]
    \itmopolygons{
        \vfill
        \Huge{Спасибо за внимание!}
        \vfill
        \includegraphics[scale=.5]{itmo/slogan.pdf}
    }
\end{frame}

\end{document}
