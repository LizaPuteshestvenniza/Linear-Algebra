\section{Задание 2 }
\plan
\begin{frame}{Условия задачи}
 \[
Q\left(x,y,z\right)=x^{2}+y^{2}+4z^{2}+2xy+4xz+4yz-6z+1=0
  \]
    \begin{block}{План}
        \begin{itemize}
            \item Привести уравнение к канонической форме методом ортогональных преобразований
            \item Нарисовать исходную поверхность и три ортонормированных вектора, которые образуют ее канонический базис
            \item Добавить линии span для удобства
        \end{itemize}
    \end{block}
\end{frame}
\subsection{}
\begin{frame}{Определение}
    Квадратичную форму с действительными коэффициентами
    \[
        Q\left(x_{1}, x_{2}, \ldots, x_{n}\right)=\sum_{i=1}^{n} \sum_{j=1}^{n} a_{i j} x_{i} x_{j}
    \]
    можно привести к каноническому виду:
    \[
    Q\left(x_{1}^{*}, x_{2}^{*}, \ldots, x_{n}^{*}\right)=\lambda_{1} x_{1}^{* 2}+\lambda_{2} x_{2}^{* 2}+\ldots+\lambda_{n} x_{n}^{* 2}
    \]
\end{frame}
\begin{frame}{Матрица A и T}
    \[
        Q(x, y, z)=x^{2}+y^{2}+4 z^{2}+2 x y+4 x z+4 y z-6 z+1
    \] 
    \[
        A=\left(\begin{array}{lll}
        1 & 1 & 2 \\
        1 & 1 & 2 \\
        2 & 2 & 4
        \end{array}\right)
    \]
    \[
        T\left(\begin{array}{l}
        x \\
        y \\
        z
        \end{array}\right)=\left(\begin{array}{l}
        x^{*} \\
        y^{*} \\
        z^{*}
        \end{array}\right)
        \]
    $T$ - ортогональная матрица перехода. \\
    $A$ - матрица квадратичной формы. 
\end{frame}
\begin{frame}{Нахождение собственных чисел}
    $\lambda_{1}, \lambda_{2}, \lambda_{3}$ - собственные числа матрицы. Столбцы матрицы $T$ являются координатами ортонормированного базиса $\left(\mathbf{e}_{1}^{*}, \mathbf{e}_{2}^{*}, \mathbf{e}_{3}^{*}\right)$, в котором матрица имеет диагональный вид.
    \[
        |A-\lambda E|=\left|\begin{array}{ccc}
        1-\lambda & 1 & 2 \\
        1 & 1-\lambda & 2 \\
        2 & 2 & 4-\lambda
        \end{array}\right|=-\lambda^{3}+6 \lambda^{2}=-(\lambda-6) \lambda^{2}=0
    \]
    \[
        \lambda_{1,2}=0, \quad \lambda_{3}=6
    \]
\end{frame}
\begin{frame}{Вычисление собственных векторов}
    $$
    \begin{aligned}
    & A \mathbf{x}=\lambda_{1,2} \mathbf{x} \\
    &\left(\begin{array}{lll}
    1 & 1 & 2 \\
    1 & 1 & 2 \\
    2 & 2 & 4
    \end{array}\right)\left(\begin{array}{l}
    x_{1} \\
    x_{2} \\
    x_{3}
    \end{array}\right)=0 \cdot\left(\begin{array}{l}
    x_{1} \\
    x_{2} \\
    x_{3}
    \end{array}\right) \rightarrow\left(\begin{array}{lll}
    1 & 1 & 2 \\
    1 & 1 & 2 \\
    2 & 2 & 4
    \end{array}\right)\left(\begin{array}{l}
    x_{1} \\
    x_{2} \\
    x_{3}
    \end{array}\right)=\left(\begin{array}{l}
    0 \\
    0 \\
    0
    \end{array}\right) \rightarrow \\
    &\left\{\begin{array}{l}
    x_{1}+x_{2}+2 x_{3}=0 \\
    x_{1}+x_{2}+2 x_{3}=0 \\
    2 x_{1}+2 x_{2}+4 x_{3}=0
    \end{array}\right. \\
    &x_{com}=\left(\begin{array}{c}-x_{2}-2 x_{3} \\ x_{2} \\ x_{3}\end{array}\right)=x_{2}\left(\begin{array}{r}-1 \\ 1 \\ 0\end{array}\right)+x_{3}\left(\begin{array}{r}-2 \\ 0 \\ 1\end{array}\right)
    \end{aligned}
    $$
\end{frame}
\begin{frame}{Вычисление собственных векторов}
\[
    \begin{aligned}
    &\bar{f}_{1}=\left(\begin{array}{r}-1 \\ 1 \\ 0\end{array}\right) \text { - первый вектор фундаментальной системы; } \\
    &\bar{f}_{2}=\left(\begin{array}{r}-2 \\ 0 \\ 1\end{array}\right) \text { - второй вектор фундаментальной системы; } \\
    &\bar{u}_{1}=\bar{f}_{1}+\mu \cdot \bar{f}_{2} \rightarrow \mu=-\frac{\overline{f_{1}} \cdot \overline{f_{2}}}{\bar{f}_{2} \cdot \overline{f_{2}}}=-\frac{2}{5} \rightarrow \bar{u}_{1}=
    \left(\begin{array}{r}-0.2 \\ 1 \\ -0.4\end{array}\right) \\
    &\bar{u}_{2}=\bar{f}_{2}=\left(\begin{array}{r}-2 \\ 0 \\ 1\end{array}\right)
    \end{aligned}
\]
\end{frame}
\begin{frame}{Вычисление собственных векторов}
\[
    \begin{aligned}
        & A \mathbf{x}=\lambda_{3} \mathbf{x}\\
        &\left(\begin{array}{lll}
        1 & 1 & 2 \\
        1 & 1 & 2 \\
        2 & 2 & 4
        \end{array}\right)\left(\begin{array}{l}
        x_{1} \\
        x_{2} \\
        x_{3}
        \end{array}\right)=6\left(\begin{array}{l}
        x_{1} \\
        x_{2} \\
        x_{3}
        \end{array}\right) \\
        &\left(\left(\begin{array}{lll}
        1 & 1 & 2 \\
        1 & 1 & 2 \\
        2 & 2 & 4
        \end{array}\right)-6\left(\begin{array}{lll}
        1 & 0 & 0 \\
        0 & 1 & 0 \\
        0 & 0 & 1
        \end{array}\right)\right)\left(\begin{array}{l}
        x_{1} \\
        x_{2} \\
        x_{3}
        \end{array}\right)=
        \left(\begin{array}{l}
        0 \\
        0 \\
        0
        \end{array}\right) \\
        &\left(\begin{array}{rrr}
        -5 & 1 & 2 \\
        1 & -5 & 2 \\
        2 & 2 & -2
        \end{array}\right)\left(\begin{array}{l}
        x_{1} \\
        x_{2} \\
        x_{3}
        \end{array}\right)=\left(\begin{array}{l}
        0 \\
        0 \\
        0
        \end{array}\right) \rightarrow 
        \left\{\begin{array}{l}
        -5 x_{1}+x_{2}+2 x_{3}=0 \\
        x_{1}-5 x_{2}+2 x_{3}=0 \\
        2 x_{1}+2 x_{2}-2 x_{3}=0
        \end{array}\right.\\
        &{x}_{com}=\left(\begin{array}{r}0.5 \\ 0.5 \\ 1\end{array}\right) \rightarrow \bar{u}_{3}=\left(\begin{array}{r}0.5 \\ 0.5 \\ 1\end{array}\right)
    \end{aligned}
\]
\end{frame}
\begin{frame}{Проверка и нормализация векторов}
    Сделаем проверку ортогональности векторов:
    \begin{align*}
        &\bar{u}_{1} \cdot \bar{u}_{2} = 0.2 \cdot 2 - 0.4 = 0 \\
        &\bar{u}_{1} \cdot \bar{u}_{3} = (-0.2) \cdot 0.5 + 0.5 - 0.4 = 0 \\
        &\bar{u}_{2} \cdot \bar{u}_{3} = -2 \cdot 0.5 + 1 = 0
    \end{align*}
    Вектора ортогональны, теперь их нужно нормализовать:
    \begin{align*}
        &e_{1}^{*}=\frac{\bar{u}_{1}}{|{\bar{u}_{1}}|}=\left(-\frac{1}{\sqrt{30}}, \frac{5}{\sqrt{30}}, -\frac{2}{\sqrt{30}}\right) \\
        &e_{2}^{*}=\frac{\bar{u}_{2}}{|{\bar{u}_{2}}|}=\left(-\frac{2}{\sqrt{5}}, 0, \frac{1}{\sqrt{5}}\right) \\
        &e_{3}^{*}=\frac{\bar{u}_{3}}{|{\bar{u}_{3}}|}=\left(\frac{1}{\sqrt{6}}, \frac{1}{\sqrt{6}}, \frac{2}{\sqrt{6}}\right)
    \end{align*}
\end{frame}
\begin{frame}{Вывод канонического уравнения}
    Следовательно матрица
   $$
    \left(\begin{array}{ccc}
    -\frac{1}{\sqrt{30}} & -\frac{2}{\sqrt{5}} & \frac{1}{\sqrt{6}} \\
    \frac{5}{\sqrt{30}} & 0 & \frac{1}{\sqrt{6}} \\
    -\frac{2}{\sqrt{30}} & \frac{1}{\sqrt{5}} & \frac{2}{\sqrt{6}}
    \end{array}\right)
    $$
    Получаем преобразование координат:
    $$
    \left\{\begin{array}{l}
    x=-\frac{1}{\sqrt{30}} x^{*} - \frac{2}{\sqrt{5}} y^{*}+\frac{1}{\sqrt{6}} z^{*} \\
    y=\frac{5}{\sqrt{30}} x^{*} + \frac{1}{\sqrt{6}} z^{*} \\
    z=-\frac{2}{\sqrt{30}} x^{*} + \frac{1}{\sqrt{5}} y^{*} + \frac{2}{\sqrt{6}} z^{*}
    \end{array}\right.
    $$
    в базисе $\left(e_{1}^{*},e_{2}^{*},e_{3}^{*}\right)$ исходная квадратичная форма принимает канонический вид:
    $$
    6 z^{* 2} + \frac{12}{\sqrt{30}}x^{*} - \frac{6}{\sqrt{5}}y^{*} -\frac{12}{\sqrt{6}}z^{*} + 1 = 0
    $$
\end{frame}
\begin{frame}{Результат}
    \begin{figure}
        \centering
        \includegraphics[scale=0.3]{fig/task-2-2/img1.png}
        \caption{Исходный параболический цилиндр с линиями векторов канонического базиса}
    \end{figure}
\end{frame}
\begin{frame}{Результат}

    \begin{figure}
        \begin{minipage}{.49\textwidth}
            \centering
            \includegraphics[scale=0.2]{fig/task-2-2/kon.jpg}
        \end{minipage}
        \begin{minipage}{.49\textwidth}
            \centering
            \includegraphics[scale=0.2]{fig/task-2-2/kon2.jpg}
        \end{minipage}
        \caption{Каноническое изображение}
    \end{figure}
\end{frame}
