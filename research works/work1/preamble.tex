\usepackage[utf8]{inputenc}
\usepackage[T1,T2A]{fontenc}
\usepackage[english,russian]{babel}
\usepackage{graphicx}
\usepackage{
amsmath, % базовый пакет для отображения математический формул
amssymb, % дополнительные математический символы
polynom % позволяет удобно работать с многочленами
}
\usepackage{cases}

% Без этой команды он иногда ругается.
\hypersetup{unicode=true}

% Чтобы адекватно работало копирование текста из полученной .pdf-ки.
% \usepackage{cmap}

% Это нужно, чтобы он называл рисунки без сокращения "рис.". Таблицы он называет без сокращения по умолчанию.
\addto\captionsrussian{\renewcommand{\figurename}{Рисунок}}

% Пакет для использования запятой в качестве десятичного разделителя. Следите, чтобы в формулах запятые стояли с пробелами, там где они запятые. Например $v = (x, y, z)$
% \usepackage{icomma}

% Это единственный пакет для библиографии, который у меня заработал с \footcite шаблоном.
% В презентациях лучше делать её руками через \footnote!
% \usepackage[style=mla]{biblatex}
% \addbibresource{references.bib}

\usepackage{ITMOtheme}

% Рисунок для титульного слайда.
\titlegraphic{\includegraphics[scale=.8]{itmo/logo_rus_vert_blue.pdf}}
