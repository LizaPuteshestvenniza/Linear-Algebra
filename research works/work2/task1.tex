\section{Задние 1}
\subsection{Формулировка задания}
\begin{frame}{Наибольшее и наименьшее значение функции\\
нескольких переменных в области} %это заголовок слайда

\begin{block}{Формулировка задания}
Через точку $D$ $(a,b,c)$ проведите плоскость так, чтобы объём тетраэдра, отсекаемого ею от
координатного трёхгранника, был бы наименьшим. Изобразите на графике для конкретной 
точки $А$.
\end{block}

\begin{block}{План:}
\begin{itemize}
    \item Изобразим на рисунке условие задачи
    \item Решим задачу аналитически
    \item Проиллюстрируем ответ
\end{itemize}
\end{block}
\end{frame}

\subsection{Выполнение задания}
\begin{frame}{Выведение формулы объема}
\begin{columns}[T]
    \begin{column}{0.48\textwidth}
    \begin{figure}
        \centering
        \includegraphics[scale=0.35]{images/tetrader.png}
        \caption{Отсеченная пирамида}
        \label{fig:pyramid}
    \end{figure}
    \end{column}
    \begin{column}{0.48\textwidth}
        Уравнение искомой плоскости: $\frac{x}{A}+\frac{y}{B}+\frac{z}{C}=1$, где $A, B$ и $C$ являются отрезками, отсекаемыми плоскостью на координатных осях. Поскольку точка $D$ $(a,b,c)$ принадлежит плоскости, то тогда: $$\frac{a}{A}+\frac{b}{B}+\frac{c}{C}=1 \Rightarrow C=\frac{c}{1-\frac{a}{A}-\frac{b}{B}}$$
        Объем тетраэдра: $$V=\frac{1}{6} A B C=\frac{1}{6} A B \frac{c}{1-\frac{a}{A}-\frac{b}{B}}$$
    \end{column}
\end{columns}
\end{frame}

\begin{frame}{Выведени уравнения плоскости}
$$
\frac{\partial V}{\partial A}=\frac{B c}{6} \frac{1-\frac{2 a}{A}-\frac{b}{B}}{\left(1-\frac{a}{A}-\frac{b}{B}\right)^{2}}; \quad\frac{\partial V}{\partial B}=\frac{A c}{6} \frac{1-\frac{a}{A}-\frac{2 b}{B}}{\left(1-\frac{a}{A}-\frac{b}{B}\right)^{2}}
$$
\Large
$$
\left\{\begin{array} { l } 
{ \frac { \partial V } { \partial A } = 0 , } \\
{ \frac { \partial V } { \partial B } = 0 }
\end{array} \Rightarrow \left\{\begin{array} { c } 
{ 1 - \frac { 2 a } { A } - \frac { b } { B } = 0 , } \\
{ 1 - \frac { a } { A } - \frac { 2 b } { B } = 0 }
\end{array} \Rightarrow \left\{\begin{array}{c}
A=3 a \\
B=3 b \\
C=3 c
\end{array}\right.\right.\right.
$$
Таким образом, получаем уравнение плоскости:
$$\frac{x}{a}+\frac{y}{b}+\frac{z}{c}=3 .$$
\end{frame}

\begin{frame}{Пример для точки E(3,2,1)}
\begin{columns}[T]
    \begin{column}{0.48\textwidth}
    \begin{figure}
        \centering
        \includegraphics[scale=0.4]{images/piramidD.png}
        \caption{Отсеченная пирамида}
        \label{fig:pyramid_E_point}
    \end{figure}
    \end{column}
    \begin{column}{0.48\textwidth}
       Получаем уравнение плоскости:
       $$\frac{x}{3}+\frac{y}{2}+\frac{z}{1}=3$$
    \end{column}
\end{columns}
\end{frame}

