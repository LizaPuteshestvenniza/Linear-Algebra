\section{Задние 3}
\subsection{Формулировка задания}
\begin{frame}{Поток векторного поля} %это заголовок слайда
\begin{columns}
\begin{column}{0.4\paperwidth}
\centering
\textbf{Формулировка задания:} \\
Дано тело $T$, ограниченное следующими поверхностями: \\
$$y + \sqrt{1 - x^{2} - z^{2}} = 0, \hspace{1mm} y + 2\sqrt{x^{2} + z^{2}} = 2$$ 
На рисунке представлено сечение тела Т координатной плоскостью $Oyz$. \\
\vspace{2mm}
\includegraphics[width=0.2\paperwidth]{images/Zadanie4.png}
\end{column}
\begin{column}{0.4\paperwidth}
\textbf{План:}
\begin{itemize}
    \item Изобразить тело Т на графике в пространстве
    \item Вычислить пототк поля $\Vec{a} = (\cos{zy})\Vec{i} + x\Vec{j} + (e^{y^{2}} - 5z)\Vec{k}$ через боковую поверхность тела $T$, образованную вращением дуги $ABC$ вокруг оси $Oy$, в направлении внешней нормали поверхности тела $T$. 
\end{itemize}
\end{column}
\end{columns}
\end{frame}

%%%%%%%%%%%%%%%

\subsection{Изображение тела $T$}
\begin{frame}{Изображение тела $T$}
\begin{block}
\centering
Тело $T$, ограниченное следующими поверхностями: \\
$$y + \sqrt{1 - x^{2} - z^{2}} = 0, \hspace{1mm} y + 2\sqrt{x^{2} + z^{2}} = 2$$ 
\centering
\includegraphics[scale=0.3]{images/Zadanie4Grafik.png}
\end{block} 
\end{frame}

%%%%%%%%%%%%%%%

\subsection{Вычисление потока поля}
\begin{frame}{Вычисление потока поля}
\begin{block}{Расчет:}
\centering
$$\Vec{a} = (\cos{zy})\Vec{i} + x\Vec{j} + (e^{y^{2}} - 5z)\Vec{k}$$
через полусферу:
$x^{2} + y^{2} + z^{2} = 1 $ при $y \leq 0$ \\
Поверхность проецируем на плоскость $Oxz$ \\
Добавим к полусфере круг:
\[
\begin{cases}
    1 = x^{2} + z^{2} \\
    0 = y
\end{cases}
\]
Найдем поток через замкнутую поверхность по теореме Остроградского-Гаусса:
$$
div = \frac{\delta P}{\delta x} + \frac{\delta Q}{\delta y} + \frac{\delta R}{\delta z} = 0 + 0 - 5 = -5
$$
\end{block}  
\end{frame}

%%%%%%%%%%%%%%%

\subsection{Вычисление потока поля}
\begin{frame}{Вычисление потока поля}
\begin{block}{Расчет:}
\centering
$$
\dot{\Pi} = \iiint_{v} div \Vec{a} dv = -5\iiint_{v} dv = 
$$
$$
= -5 \cdot \frac{1}{2} \cdot \frac{4\pi}{3} = \frac{-10\pi}{3}
$$
Найдем поток через круг $y = 0$, поэтому $n\textdegree = (0; 1; 0)$ \\
$(a \cdot n\textdegree) = x$
$$
x^{2} + z^{2} \leq 1
$$
$$
\Pi_{1} = \iint_{s} xds = \int_{-1}^{1} dx \int_{-\sqrt{1-x^{2}}} ^{sqrt{1-x^{2}}} xdz = \int_{-1} ^{1} xdx \bigg( z\bigg|_{-\sqrt{1-x^{2}}} ^{sqrt{1-x^{2}}} \bigg) =
$$
\end{block}  
\end{frame}

%%%%%%%%%%%%%%%

\subsection{Вычисление потока поля}
\begin{frame}{Вычисление потока поля}
\begin{block}{Расчет:}
\centering
$$
= \int_{-1} ^{1} 2\sqrt{1-x^{2}}xdx = - \int_{-1}^{1} \sqrt{1-x^{2}} d(1-x^{2}) =
$$
$$
= - \frac{(1-x^{2})^{\frac{3}{2}}}{\frac{3}{2}}\bigg|_{-1}^{1} = 0
$$
$$\Pi = \Pi_1 + \Pi_2$$ 
где $\Pi_{2}$ - поток полусф $\implies \Pi_2 = - \frac{10\pi}{3} $
\end{block}  
\end{frame}
