\section{Задние 3}
\subsection{Формулировка задания}
\begin{frame}{Потенциал векторного поля} %это заголовок слайда
\begin{block}{Формулировка задания:}
\centering
Дано векторное поле $\vec{H}$: \\
$(y\cos{xy};x\cos{xy});$
\end{block}
\begin{block}{План:}
\begin{itemize}
    \item Убедиться, что поле потенциально
    \item Найти уравнения векторных линий
    \item Изобразить векторные линии на рисунке
    \item Найти потенциал поля при помощи криволинейного интеграла
    \item Изобразить эквипотенциальные линии. Проиллюстрировать ортогональность линий уровня и векторных линий. 
    \item Зафиксировать точки $A$ и $B$ на какой-либо векторной линии. Вычислить работу поля вдоль этой линии. 
\end{itemize}
\end{block}
\end{frame}

%%%%%%%%%%%%%%%

\subsection{Условие потенциальности}
\begin{frame}{Условие потенциальности}
\begin{block}{Проверим условие потенциальности:}
\begin{align*}
      Q = y\cos{xy} \hspace{3mm} P = x\cos{xy} \\
      \frac{\delta P}{\delta y} = \cos{xy} - yx\sin{xy} \\
      \frac{\delta Q}{\delta x} = \cos{xy} - yx\sin{xy} \\
\end{align*}
\centering
так как $\frac{\delta P}{\delta y} =  \frac{\delta Q}{\delta x}$, то поле потенциальное. 
\end{block}
\end{frame}

%%%%%%%%%%%%%%%

\subsection{Векторные линии}
\begin{frame}{Векторные линии}
\begin{columns}
\begin{column}{0.4\paperwidth}
\centering
Найдем уравнения векторных линий и изобразим их. \\
$$
     \frac{dx}{P} = \frac{dy}{Q} 
$$
$$
   \frac{dx}{y\cos{xy}} = \frac{dy}{x\cos{xy}} \implies \frac{dx}{y} = \frac{dy}{x}
$$
$$
xdx = ydy 
$$
$$
x^{2} = y^{2} + C 
$$
$$
x^{2} - y^{2} = C
$$
\end{column}
\begin{column}{0.4\paperwidth}
\centering
Гиперболы: 
\vspace{5mm}
\includegraphics[width=0.4\paperwidth]{images/GrafikGiperboli.png}
\end{column}
\end{columns}
\end{frame}

%%%%%%%%%%%%%%%

\subsection{Потенциал}
\begin{frame}{Потенциал} 
\begin{block}
\centering
Найдем потенциал поля при помощи криволинейного интеграла:
$$
U = \int_{x_0} ^x P(x, y_0)dx + \int_{y_0} ^y Q(x, y)dy
$$
\centering
Пусть $x_0 = y_0 = 0$, тогда 
$$
U = \int_0^x 0dx + \int_0^y x\cos{xy}dy = \sin{xy} + C
$$
\end{block}
\end{frame}

%%%%%%%%%%%%%%%

\subsection{Линии уровня}
\begin{frame}{Линии уровня}
\begin{columns}
\begin{column}{0.4\paperwidth}
\centering
Изобразим эквипотенциальные линии, проиллюстрировав ортогональность линий уровня и векторных линий.  
$$
\sin{xy} = C
$$
$$
xy = \arcsin{C}
$$
$$
y = \frac{\arcsin{C}}{x}
$$
\end{column}
\begin{column}{0.4\paperwidth}
\centering
Гиперболы:
\vspace{5mm}
\includegraphics[width=0.4\paperwidth]{images/GrafikOrtGiperb3Zadanie.png}
\end{column}
\end{columns}
\end{frame}

%%%%%%%%%%%%%%%

\subsection{Линии уровня}
\begin{frame}{Линии уровня}
\begin{columns}
\begin{column}{0.4\paperwidth}
\centering
Строим:
\vspace{5mm}
\includegraphics[width=0.4\paperwidth]{images/Grafik3ZadanieNov.png}
\end{column}
\begin{column}{0.4\paperwidth}
\centering
\vspace{20mm} \\
Пусть C = 0 \\
векторные линии: $x^2 = y^2$ \\
Линии уровня: $xy = 0$
\end{column}
\end{columns}
\end{frame}

\subsection{Работа поля}
\begin{frame}{Работа поля}
\begin{columns}
\begin{column}{0.4\paperwidth}
\centering 
 При C = 1 
$$
x^2 - y^2 = 1
$$
\centering
Пусть $A(1;0)$\hspace{1mm} $B(2;\sqrt{3})$ \\
$$
A = U(B) - U(A) = 
$$
$$
= \sin{2\sqrt{3}} - \sin{0} = \sin{2\sqrt{3}}
$$
В общем виде: \\
$A(x_0; \sqrt{x_0^{2} - 1})$ \hspace{1mm} $B(x_1; \sqrt{x_1^{2} - 1})$
$$
A = \sin{x_1\sqrt{x_1^{2} - 1}} - \sin{x_0\sqrt{x_0^{2} - 1}}
$$
\end{column}
\end{columns}
\end{frame}
